
%\renewcommand\year{2017}
%%%%%%%%%%%%%%%%%%%%%%%%%%%%%%%%%%%%%%%%%%%%%%%%%%%%%%%%%%%%%%%%%%%%%%%%%%%%%%
% Logo da PUC Minas
%%%%%%%%%%%%%%%%%%%%%%%%%%%%%%%%%%%%%%%%%%%%%%%%%%%%%%%%%%%%%%%%%%%%%%%%%%%%%%
\newcommand{\logoPUC}[1]{
\lhead{\begin{tabular*}{18.0cm}{p{2cm}p{14cm}@{\extracolsep{\fill}}}
 \multirow{3}{2.4cm}{\epsfig{file=#1,width=1.5cm}}
   & \\
   &  \multicolumn{1}{c}{\usefont{T1}{cmss}{bx}{n} \UNIVERSIDADE} \\
   &  \multicolumn{1}{c}{\usefont{T1}{cmss}{x}{n} \UNIDADE $~-~$ \INSTITUTO $~-~$ \CURSO }\\
\end{tabular*}
}}


%%%%%%%%%%%%%%%%%%%%%%%%%%%%%%%%%%%%%%%%%%%%%%%%%%%%%%%%%%%%%%%%%%%%%%%%%%%%%%
% Caracteristicas da disciplina
%%%%%%%%%%%%%%%%%%%%%%%%%%%%%%%%%%%%%%%%%%%%%%%%%%%%%%%%%%%%%%%%%%%%%%%%%%%%%%

%\dadosDisciplina{\DISCIPLINA}{\CURSO}{\TURNO}{\ANO}{\SEMESTRE}{\PROFESSOR}{\EMAILPROFESSOR}

\newcommand{\dadosDisciplina}[7]{\noindent
\resizebox{.95\textwidth}{!}{
\begin{tabular*}{18.5cm}{|p{8.6cm}@{\extracolsep{\fill}}|p{4.5cm}|p{1.7cm}|}
\hline

\sf Disciplina & \sf \hspace{-1cm} Curso  & \hspace{-1cm} \sf Semestre \\

\multicolumn{1}{|r|}{#1} &
\multicolumn{1}{r|}{#2} &
\multicolumn{1}{r|}{#4/#5$^\circ$} \\ \hline \hline
\multicolumn{3}{|l|}{\sf Professor} \\
\multicolumn{3}{|c|}{#6~(#7)} \\ \hline
\end{tabular*}
}
}


%%%%%%%%%%
%%%%%
%%%%%%%%%%

\newcommand\dadosAvancadosDisciplina[9]
{
    \def\tempa{#1}%
    \def\tempb{#2}%
    \def\tempc{#3}%
    \def\tempd{#4}%
    \def\tempe{#5}%
    \def\tempf{#6}%
    \def\tempg{#7}
    \def\temph{#8}%
    \def\tempi{#9}%
    \dadosAvancadosDisciplinaCont
}
\newcommand{\dadosAvancadosDisciplinaCont}[2]{
\noindent
\resizebox{.95\textwidth}{!}{
\begin{tabular*}{18.5cm}{|p{3cm}<{\centering}@{\extracolsep{\fill}}|p{3.5cm}<{\centering}|p{3cm}<{\centering}|p{5cm}<{\centering}@{\extracolsep{\fill}}|p{4.7cm}<{\centering}|}
\hline
%\multicolumn{3}{|c|}{\sf Carga Horária Semanal} & \sf Carga Horária Semestral & \sf Número de Créditos \\
%\hline
%\sf Teórica & \sf Prática & \sf Total &  &  \\
%\tempa & \tempb & \tempc & \tempd\textbf{} & \tempe \\
%\hline
\multicolumn{5}{|l|}{\sf Objetivos} \\
\multicolumn{5}{|l|}{\hspace{0.3cm} \begin{minipage}{17cm}
\vspace{0.3cm}
\input{\tempf}
\vspace{0.3cm}
\end{minipage}}\\ \hline

\multicolumn{5}{|l|}{\sf Ementa} \\
\multicolumn{5}{|l|}{\hspace{0.3cm} \begin{minipage}{17cm}
\vspace{0.3cm}
\input{\tempg}
\vspace{0.3cm}
\end{minipage}}\\ \hline
%\end{tabular*}


\multicolumn{5}{|l|}{\sf Processo de Avaliação} \\ %\hline
%\multicolumn{4}{|l}{\hspace{0.3cm} Provas}     & \temph \\
%\multicolumn{4}{|l}{\hspace{0.3cm} Trabalhos Práticos} &  \tempi \\
%\multicolumn{4}{|l}{\hspace{0.3cm} Total}              & 100 \\


%\hline

%\multicolumn{5}{|l|}{\sf Trabalhos Práticos} \\
\multicolumn{5}{|l|}{\hspace{0.3cm} \begin{minipage}{17.6cm}
\vspace{0.3cm}
\input{#1}
\vspace{0.3cm}
\end{minipage}}\\ \hline
\multicolumn{5}{|l|}{\sf Bibliografia} \\
\multicolumn{5}{|l|}{\hspace{0.3cm} \begin{minipage}{17cm}
\vspace{0.3cm} \small
\input{#2}
\vspace{0.3cm}
\end{minipage}}\\ \hline

\end{tabular*}
}
}


\newcommand{\tikzcalhead}[1]{\node[anchor=mid](seg){S};
\node[base right=of seg](ter){T}; \node[base right=of ter](qua){Q};
\node[base right=of qua](qui){Q}; \node[base right=of qui](sex){S};
\node[base right=of sex](sab){S}; \node[darkgreen,base right=of sab](dom){D};
\node[darkgreen,above=of qui]{\textbf{#1}};}

%%%%%%%%%%%%%%%%%%%%%%%%%%%%%%%%%%%%%%%%%%%%%%%%%%%%%%%%%%%%%%%%%%%%%%%%%%%%%%
% Calendário Tikz Colorido
%%%%%%%%%%%%%%%%%%%%%%%%%%%%%%%%%%%%%%%%%%%%%%%%%%%%%%%%%%%%%%%%%%%%%%%%%%%%%%

%% Definicao de Cores
\newcommand{\coraula}{blue}
\newcommand{\cordomingo}{gray}

\newcommand{\markprova}{\node [fill=yellow!50!white,circle, minimum size=.5cm] {};}
\newcommand{\markexerc}{\node [fill=green!50!white,circle, minimum size=.5cm] {};}
\newcommand{\markferiado}{\node [fill=red!35!white,circle, minimum size=.5cm] {};}


\newcommand{\markdiasferiados}{
	%	if (equals=02-15) {\node [draw,regular polygon,regular polygon sides=3] {};}
	%	if (equals=02-20) {\node [draw,cloud] {};}
	%
	%   Marcação de Feriados
	%
	%if (equals=02-01) {\markferiado}
	%if (equals=02-02) {\markferiado}
	if (equals=02-18) {\markferiado}
	if (equals=02-19) {\markferiado}
	if (equals=02-20) {\markferiado}
	if (equals=02-21) {\markferiado}
	if (equals=02-22) {\markferiado}
	%if (equals=03-02) {\markferiado}
	%if (equals=03-03) {\markferiado}
	%if (equals=03-04) {\markferiado}
	%if (equals=03-05) {\markferiado}
	%if (equals=03-06) {\markferiado}
	%if (equals=03-31) {\markferiado}
	if (equals=04-06) {\markferiado}
	if (equals=04-07) {\markferiado}
	if (equals=04-08) {\markferiado}
	if (equals=04-09) {\markferiado}
	%if (equals=04-12) {\markferiado}
	%if (equals=04-20) {\markferiado}
	if (equals=04-21) {\markferiado}
	%if (equals=04-30) {\markferiado}
	if (equals=05-01) {\markferiado}
	%if (equals=05-02) {\markferiado}
	%if (equals=06-01) {\markferiado}
	if (equals=06-08) {\markferiado}
	if (equals=08-15) {\markferiado}
	if (equals=09-07) {\markferiado}
	%if (equals=09-08) {\markferiado}
	if (equals=10-12) {\markferiado}
	if (equals=10-13) {\markferiado}
	if (equals=10-14) {\markferiado}
	%if (equals=10-15) {\markferiado}
	if (equals=11-02) {\markferiado}
	%if (equals=11-03) {\markferiado}
	if (equals=11-15) {\markferiado}
	if (equals=12-08) {\markferiado}
}


%% Devem ser editados de acordo com disciplina!!!%%%%%%%%

\newcommand{\markdiasexerc}{
	%
	%   Marcação de Exercicios
	%
%	if (equals=09-05) {\markexerc}
%	if (equals=09-16) {\markexerc}
%	if (equals=11-06) {\markexerc}
%	if (equals=12-02) {\markexerc}
}



\newcommand{\markdiasprova}{
	%
	%   Marcação de Provas
	%
%	if (equals=04-03) {\markprova}
%	if (equals=05-20) {\markprova}
%	if (equals=06-26) {\markprova}
%	if (equals=07-03) {\markprova}
}

%%%%%%%%%%%%%%%%%%%%%%%%%%%%%%%%%%%%%%%%%%%%%%%%%%%%%%%%%

\newcommand{\calVisualColorido}[5]{
\noindent
\resizebox{.95\textwidth}{!}{
\ifnum#1=1 %
%\begin{tikzpicture}[ampersand replacement=\&, every calendar/.style={week list}]
%\matrix[row 1/.style={ampersand replacement=\&, node distance=.3ex}, row 3/.style={node distance=.3ex}, column sep=1ex,%
%draw=darkgreen,thick,rounded corners=5pt,%
%postaction={decorate,decoration={markings,mark=at position 0.51 with
\begin{tikzpicture}[ampersand replacement=\&, every calendar/.style={week list},every day/.style={anchor=mid},day text=\%d0]
\matrix[row 1/.style={ampersand replacement=\&, node distance=.3ex}, row 3/.style={node distance=.3ex}, column sep=1ex,%
draw=darkgreen,thick,rounded corners=5pt,%
postaction={decorate,decoration={markings,mark=at position 0.20 with
        {\node[fill=white,text=darkgreen,font={\bfseries\Large}] (#4) {#4};}}} % year on frame: decorations.markings library
]{%
    \tikzcalhead{Fevereiro}  \&
    \tikzcalhead{Março} \&
    \tikzcalhead{Abril}  \&
    \tikzcalhead{Maio}  \&
   \tikzcalhead{Junho} \\
    \calendar(fev)[dates=#4-02-01 to #4-02-last, if={(Sunday) [\cordomingo]},if={(#2) [\coraula]},if={(#3) [\coraula]}] \markdiasferiados \markdiasexerc \markdiasprova; \&
    \calendar(mar)[dates=#4-03-01 to #4-03-last, if={(Sunday) [\cordomingo]},if={(#2) [\coraula]},if={(#3) [\coraula]}] \markdiasferiados \markdiasexerc \markdiasprova; \&
    \calendar(abr)[dates=#4-04-01 to #4-04-last, if={(Sunday) [\cordomingo]},if={(#2) [\coraula]},if={(#3) [\coraula]}] \markdiasferiados \markdiasexerc \markdiasprova; \&
    \calendar(mai)[dates=#4-05-01 to #4-05-last, if={(Sunday) [\cordomingo]},if={(#2) [\coraula]},if={(#3) [\coraula]}] \markdiasferiados \markdiasexerc \markdiasprova; \&
    \calendar(jun)[dates=#4-06-01 to #4-06-last,if={(Sunday) [\cordomingo]},if={(#2) [\coraula]},if={(#3) [\coraula]}, if={(at least=#5) [black]}] \markdiasferiados \markdiasexerc \markdiasprova;  \\
};
\end{tikzpicture}
\else%
\begin{tikzpicture}[ampersand replacement=\&, every calendar/.style={week list},every day/.style={anchor=mid},day text=\%d0]
\matrix[row 1/.style={ampersand replacement=\&, node distance=.3ex}, row 3/.style={node distance=.3ex}, column sep=1ex,%
draw=darkgreen,thick,rounded corners=5pt,%
postaction={decorate,decoration={markings,mark=at position 0.20 with
        {\node[fill=white,text=darkgreen,font={\bfseries\Large}] (#4) {#4};}}} % year on frame: decorations.markings library
]{%
    \tikzcalhead{Agosto}   \&
    \tikzcalhead{Setembro} \&
    \tikzcalhead{Outubro}  \&
    \tikzcalhead{Novembro} \&
    \tikzcalhead{Dezembro} \\
    \calendar(ago)[dates=#4-08-01 to #4-08-last, if={(Sunday) [\cordomingo]},if={(#2) [\coraula]},if={(#3) [\coraula]}]  \markdiasferiados \markdiasexerc \markdiasprova; \&
    \calendar(set)[dates=#4-09-01 to #4-09-last, if={(Sunday) [\cordomingo]},if={(#2) [\coraula]},if={(#3) [\coraula]}]  \markdiasferiados \markdiasexerc \markdiasprova; \&
    \calendar(out)[dates=#4-10-01 to #4-10-last, if={(Sunday) [\cordomingo]},if={(#2) [\coraula]},if={(#3) [\coraula]}]  \markdiasferiados \markdiasexerc \markdiasprova; \&
    \calendar(nov)[dates=#4-11-01 to #4-11-last, if={(Sunday) [\cordomingo]},if={(#2) [\coraula]},if={(#3) [\coraula]}]  \markdiasferiados \markdiasexerc \markdiasprova; \&
    \calendar(dez)[dates=#4-12-01 to #4-12-last, if={(Sunday) [\cordomingo]},if={(#2) [\coraula]},if={(#3) [\coraula]}, if={(at least=#5) [black]}]  \markdiasferiados \markdiasexerc \markdiasprova; \\
};
\end{tikzpicture}
\fi
}
}


%%%%%%%%%%%%%%%%%%%%%%%%%%%%%%%%%%%%%%%%%%%%%%%%%%%%%%%%%%%%%%%%%%%%%%%%%%%%%%
% Calendário Tikz Colorido Estendido para 3 vezes por semana
%%%%%%%%%%%%%%%%%%%%%%%%%%%%%%%%%%%%%%%%%%%%%%%%%%%%%%%%%%%%%%%%%%%%%%%%%%%%%%

\newcommand{\calVisualTresColorido}[6]{
	\noindent
	\resizebox{.95\textwidth}{!}{
		\ifnum#1=1 %
		%\begin{tikzpicture}[ampersand replacement=\&, every calendar/.style={week list}]
		%\matrix[row 1/.style={ampersand replacement=\&, node distance=.3ex}, row 3/.style={node distance=.3ex}, column sep=1ex,%
		%draw=darkgreen,thick,rounded corners=5pt,%
		%postaction={decorate,decoration={markings,mark=at position 0.51 with
		\begin{tikzpicture}[ampersand replacement=\&, every calendar/.style={week list},every day/.style={anchor=mid},day text=\%d0]
		\matrix[row 1/.style={ampersand replacement=\&, node distance=.3ex}, row 3/.style={node distance=.3ex}, column sep=1ex,%
		draw=darkgreen,thick,rounded corners=5pt,%
		postaction={decorate,decoration={markings,mark=at position 0.20 with
				{\node[fill=white,text=darkgreen,font={\bfseries\Large}] (#5) {#5};}}} % year on frame: decorations.markings library
		]{%
			\tikzcalhead{Fevereiro}  \&
			\tikzcalhead{Março} \&
			\tikzcalhead{Abril}  \&
			\tikzcalhead{Maio}  \&
			\tikzcalhead{Junho} \\
			\calendar(fev)[dates=#5-02-01 to #5-02-last, if={(Sunday) [\cordomingo]},if={(#2) [\coraula]},if={(#3) [\coraula]},if={(#4) [\coraula]}] \markdiasferiados \markdiasexerc \markdiasprova; \&
			\calendar(mar)[dates=#5-03-01 to #5-03-last, if={(Sunday) [\cordomingo]},if={(#2) [\coraula]},if={(#3) [\coraula]},if={(#4) [\coraula]}] \markdiasferiados \markdiasexerc \markdiasprova; \&
			\calendar(abr)[dates=#5-04-01 to #5-04-last, if={(Sunday) [\cordomingo]},if={(#2) [\coraula]},if={(#3) [\coraula]},if={(#4) [\coraula]}] \markdiasferiados \markdiasexerc \markdiasprova; \&
			\calendar(mai)[dates=#5-05-01 to #5-05-last, if={(Sunday) [\cordomingo]},if={(#2) [\coraula]},if={(#3) [\coraula]},if={(#4) [\coraula]}] \markdiasferiados \markdiasexerc \markdiasprova; \&
			\calendar(jun)[dates=#5-06-01 to #5-06-last, if={(Sunday) [\cordomingo]},if={(#2) [\coraula]},if={(#3) [\coraula]},if={(#4) [\coraula]}, if={(at least=#6) [black]}] \markdiasferiados \markdiasexerc \markdiasprova;  \\
		};
		\end{tikzpicture}
		\else%
		\begin{tikzpicture}[ampersand replacement=\&, every calendar/.style={week list},every day/.style={anchor=mid},day text=\%d0]
		\matrix[row 1/.style={ampersand replacement=\&, node distance=.3ex}, row 3/.style={node distance=.3ex}, column sep=1ex,%
		draw=darkgreen,thick,rounded corners=5pt,%
		postaction={decorate,decoration={markings,mark=at position 0.20 with
				{\node[fill=white,text=darkgreen,font={\bfseries\Large}] (#5) {#5};}}} % year on frame: decorations.markings library
		]{%
			\tikzcalhead{Agosto}   \&
			\tikzcalhead{Setembro} \&
			\tikzcalhead{Outubro}  \&
			\tikzcalhead{Novembro} \&
			\tikzcalhead{Dezembro} \\
			\calendar(ago)[dates=#5-08-01 to #5-08-last, if={(Sunday) [\cordomingo]},if={(#2) [\coraula]},if={(#3) [\coraula]},if={(#4) [\coraula]}] \markdiasferiados \markdiasexerc \markdiasprova; \&
			\calendar(set)[dates=#5-09-01 to #5-09-last, if={(Sunday) [\cordomingo]},if={(#2) [\coraula]},if={(#3) [\coraula]},if={(#4) [\coraula]}] \markdiasferiados \markdiasexerc \markdiasprova; \&
			\calendar(out)[dates=#5-10-01 to #5-10-last, if={(Sunday) [\cordomingo]},if={(#2) [\coraula]},if={(#3) [\coraula]},if={(#4) [\coraula]}] \markdiasferiados \markdiasexerc \markdiasprova; \&
			\calendar(nov)[dates=#5-11-01 to #5-11-last, if={(Sunday) [\cordomingo]},if={(#2) [\coraula]},if={(#3) [\coraula]},if={(#4) [\coraula]}] \markdiasferiados \markdiasexerc \markdiasprova; \&
			\calendar(dez)[dates=#5-12-01 to #5-12-last, if={(Sunday) [\cordomingo]},if={(#2) [\coraula]},if={(#3) [\coraula]},if={(#4) [\coraula]}, if={(at least=#6) [black]}] \markdiasferiados \markdiasexerc \markdiasprova; \\
		};
		\end{tikzpicture}
		\fi
	}
}

%%%%%%%%%%%%%%%%%%%%%%%%%%%%%%%%%%%%%%%%%%%%%%%%%%%%%%%%%%%%%%%%%%%%%%%%%%%%%%
% Calendário Tikz
%%%%%%%%%%%%%%%%%%%%%%%%%%%%%%%%%%%%%%%%%%%%%%%%%%%%%%%%%%%%%%%%%%%%%%%%%%%%%%
\newcommand{\calVisual}[4]{
	\noindent
	\resizebox{.95\textwidth}{!}{
		\ifnum#1=1 %
		%\begin{tikzpicture}[ampersand replacement=\&, every calendar/.style={week list}]
		%\matrix[row 1/.style={ampersand replacement=\&, node distance=.3ex}, row 3/.style={node distance=.3ex}, column sep=1ex,%
		%draw=darkgreen,thick,rounded corners=5pt,%
		%postaction={decorate,decoration={markings,mark=at position 0.51 with
		\begin{tikzpicture}[ampersand replacement=\&, every calendar/.style={week list}]
		\matrix[row 1/.style={ampersand replacement=\&, node distance=.3ex}, row 3/.style={node distance=.3ex}, column sep=1ex,%
		draw=darkgreen,thick,rounded corners=5pt,%
		postaction={decorate,decoration={markings,mark=at position 0.20 with
				{\node[fill=white,text=darkgreen,font={\bfseries\Large}] (#4) {#4};}}} % year on frame: decorations.markings library
		]{%
			\tikzcalhead{Fevereiro}  \&
			\tikzcalhead{Março} \&
			\tikzcalhead{Abril}  \&
			\tikzcalhead{Maio}  \&
			\tikzcalhead{Junho} \\
			\calendar(fev)[dates=#4-02-01 to #4-02-last, if={(Sunday) [darkgreen]},if={(#2) [darkred]},if={(#3) [darkred]}]; \&
			\calendar(mar)[dates=#4-03-01 to #4-03-last, if={(Sunday) [darkgreen]},if={(#2) [darkred]},if={(#3) [darkred]}]; \&
			\calendar(abr)[dates=#4-04-01 to #4-04-last, if={(Sunday) [darkgreen]},if={(#2) [darkred]},if={(#3) [darkred]}]; \&
			\calendar(mai)[dates=#4-05-01 to #4-05-last, if={(Sunday) [darkgreen]},if={(#2) [darkred]},if={(#3) [darkred]}]; \&
			\calendar(jun)[dates=#4-06-01 to #4-06-last, if={(Sunday) [darkgreen]},if={(#2) [darkred]},if={(#3) [darkred]}];  \\
		};
		\end{tikzpicture}
		\else%
		\begin{tikzpicture}[ampersand replacement=\&, every calendar/.style={week list}]
		\matrix[row 1/.style={ampersand replacement=\&, node distance=.3ex}, row 3/.style={node distance=.3ex}, column sep=1ex,%
		draw=darkgreen,thick,rounded corners=5pt,%
		postaction={decorate,decoration={markings,mark=at position 0.20 with
				{\node[fill=white,text=darkgreen,font={\bfseries\Large}] (#4) {#4};}}} % year on frame: decorations.markings library
		]{%
			\tikzcalhead{Agosto}   \&
			\tikzcalhead{Setembro} \&
			\tikzcalhead{Outubro}  \&
			\tikzcalhead{Novembro} \&
			\tikzcalhead{Dezembro} \\
			\calendar(ago)[dates=#4-08-01 to #4-08-last, if={(Sunday) [darkgreen]},if={(#2) [darkred]},if={(#3) [darkred]}]; \&
			\calendar(set)[dates=#4-09-01 to #4-09-last, if={(Sunday) [darkgreen]},if={(#2) [darkred]},if={(#3) [darkred]}]; \&
			\calendar(out)[dates=#4-10-01 to #4-10-last, if={(Sunday) [darkgreen]},if={(#2) [darkred]},if={(#3) [darkred]}]; \&
			\calendar(nov)[dates=#4-11-01 to #4-11-last, if={(Sunday) [darkgreen]},if={(#2) [darkred]},if={(#3) [darkred]}]; \&
			\calendar(dez)[dates=#4-12-01 to #4-12-last, if={(Sunday) [darkgreen]},if={(#2) [darkred]},if={(#3) [darkred]}]; \\
		};
		\end{tikzpicture}
		\fi
	}
}


%%%%%%%%%%%%%%%%%%%%%%%%%%%%%%%%%%%%%%%%%%%%%%%%%%%%%%%%%%%%%%%%%%%%%%%%%%%%%%
% Calendário Tikz Estendido para 3 vezes por semana
%%%%%%%%%%%%%%%%%%%%%%%%%%%%%%%%%%%%%%%%%%%%%%%%%%%%%%%%%%%%%%%%%%%%%%%%%%%%%%

\newcommand{\calVisualTres}[5]{
	\noindent
	\resizebox{.95\textwidth}{!}{
		\ifnum#1=1 %
		%\begin{tikzpicture}[ampersand replacement=\&, every calendar/.style={week list}]
		%\matrix[row 1/.style={ampersand replacement=\&, node distance=.3ex}, row 3/.style={node distance=.3ex}, column sep=1ex,%
		%draw=darkgreen,thick,rounded corners=5pt,%
		%postaction={decorate,decoration={markings,mark=at position 0.51 with
		\begin{tikzpicture}[ampersand replacement=\&, every calendar/.style={week list},every day/.style={anchor=mid}]
		\matrix[row 1/.style={ampersand replacement=\&, node distance=.3ex}, row 3/.style={node distance=.3ex}, column sep=1ex,%
		draw=darkgreen,thick,rounded corners=5pt,%
		postaction={decorate,decoration={markings,mark=at position 0.20 with
				{\node[fill=white,text=darkgreen,font={\bfseries\Large}] (#5) {#5};}}} % year on frame: decorations.markings library
		]{%
			\tikzcalhead{Fevereiro}  \&
			\tikzcalhead{Março} \&
			\tikzcalhead{Abril}  \&
			\tikzcalhead{Maio}  \&
			\tikzcalhead{Junho} \\
			\calendar(fev)[dates=#5-02-01 to #5-02-last, if={(Sunday) [darkgreen]},if={(#2) [darkred]},if={(#3) [darkred]},if={(#4) [darkred]}]; \&
			\calendar(mar)[dates=#5-03-01 to #5-03-last, if={(Sunday) [darkgreen]},if={(#2) [darkred]},if={(#3) [darkred]},if={(#4) [darkred]}]; \&
			\calendar(abr)[dates=#5-04-01 to #5-04-last, if={(Sunday) [darkgreen]},if={(#2) [darkred]},if={(#3) [darkred]},if={(#4) [darkred]}]; \&
			\calendar(mai)[dates=#5-05-01 to #5-05-last, if={(Sunday) [darkgreen]},if={(#2) [darkred]},if={(#3) [darkred]},if={(#4) [darkred]}]; \&
			\calendar(jun)[dates=#5-06-01 to #5-06-last, if={(Sunday) [darkgreen]},if={(#2) [darkred]},if={(#3) [darkred]},if={(#4) [darkred]}];  \\
		};
		\end{tikzpicture}
		\else%
		\begin{tikzpicture}[ampersand replacement=\&, every calendar/.style={week list}]
		\matrix[row 1/.style={ampersand replacement=\&, node distance=.3ex}, row 3/.style={node distance=.3ex}, column sep=1ex,%
		draw=darkgreen,thick,rounded corners=5pt,%
		postaction={decorate,decoration={markings,mark=at position 0.20 with
				{\node[fill=white,text=darkgreen,font={\bfseries\Large}] (#5) {#5};}}} % year on frame: decorations.markings library
		]{%
			\tikzcalhead{Agosto}   \&
			\tikzcalhead{Setembro} \&
			\tikzcalhead{Outubro}  \&
			\tikzcalhead{Novembro} \&
			\tikzcalhead{Dezembro} \\
			\calendar(ago)[dates=#5-08-01 to #5-08-last, if={(Sunday) [darkgreen]},if={(#2) [darkred]},if={(#3) [darkred]},if={(#4) [darkred]}]; \&
			\calendar(set)[dates=#5-09-01 to #5-09-last, if={(Sunday) [darkgreen]},if={(#2) [darkred]},if={(#3) [darkred]},if={(#4) [darkred]}]; \&
			\calendar(out)[dates=#5-10-01 to #5-10-last, if={(Sunday) [darkgreen]},if={(#2) [darkred]},if={(#3) [darkred]},if={(#4) [darkred]}]; \&
			\calendar(nov)[dates=#5-11-01 to #5-11-last, if={(Sunday) [darkgreen]},if={(#2) [darkred]},if={(#3) [darkred]},if={(#4) [darkred]}]; \&
			\calendar(dez)[dates=#5-12-01 to #5-12-last, if={(Sunday) [darkgreen]},if={(#2) [darkred]},if={(#3) [darkred]},if={(#4) [darkred]}]; \\
		};
		\end{tikzpicture}
		\fi
	}
}



%%%%%%%%%%%%%%%%%%%%%%%%%%%%%%%%%%%%%%%%%%%%%%%%%%%%%%%%%%%%%%%%%%%%%%%%%%%%%%
% Preambulo - Termcal
%%%%%%%%%%%%%%%%%%%%%%%%%%%%%%%%%%%%%%%%%%%%%%%%%%%%%%%%%%%%%%%%%%%%%%%%%%%%%%
\newcommand{\aviso}[2]{%
	\caltext{#1}{\vbox{\vspace{1mm}\scriptsize \color{darkgreen} \textit{#2} \vspace{1mm}}}
}

\newcommand{\feriado}[2]{%
	\options{#1}{\noclassday}
	\caltext{#1}{\cellcolor{red!40} \centering \bf \sf #2}
}

\newcommand\Segunda{Monday}
\newcommand\Terca{Tuesday}
\newcommand\Quarta{Wednesday}
\newcommand\Quinta{Thursday}
\newcommand\Sexta{Friday}
\newcommand\Sabado{Saturday}


\newcommand{\prova}[1]{%
	\caltextnext{\cellcolor{yellow!40} \centering \bf \sf #1}
}
\newcommand{\seminario}[1]{%
	\caltextnext{\cellcolor{green!20} \centering \bf \sf #1}
}
\newcommand{\exercicios}[1]{%
	\caltextnext{\cellcolor{green!20} \centering \bf \sf #1}
}

\newcommand{\semanaSeg}{%
    \calday[Segunda-feira]{\classday}
    \skipday % Terça-feira (sem aulas)
    \skipday % Quarta-feira (sem aulas)
    \skipday % Quinta-feira (sem aulas)
    \skipday % Sexta-feira (sem aulas)
    \skipday\skipday % fim de semana (sem aulas)
}

\newcommand{\semanaSegTer}{%
	\calday[Segunda-feira]{\classday}
	\calday[Terça-feira]{\classday}
	\skipday % Quarta-feira (sem aulas)
	\skipday % Quinta-feira (sem aulas)
	\skipday % Sexta-feira (sem aulas)
	\skipday\skipday % fim de semana (sem aulas)
}

\newcommand{\semanaSegQua}{%
	\calday[Segunda-feira]{\classday}
	\skipday % Terça-feira (sem aulas)
	\calday[Quarta-feira]{\classday}
	\skipday % Quinta-feira (sem aulas)
	\skipday % Sexta-feira (sem aulas)
	\skipday\skipday % fim de semana (sem aulas)
}


\newcommand{\semanaSegQui}{%
	\calday[Segunda-feira]{\classday}
	\skipday % Terça-feira (sem aulas)
	\skipday % Quarta-feira (sem aulas)
	\calday[Quinta-feira]{\classday}
	\skipday % Sexta-feira (sem aulas)
	\skipday\skipday % fim de semana (sem aulas)
}


\newcommand{\semanaSegSex}{%
	\calday[Segunda-feira]{\classday}
	\skipday % Terça-feira (sem aulas)
	\skipday % Quarta-feira (sem aulas)
	\skipday % Quinta-feira (sem aulas)
	\calday[Sexta-feira]{\classday}
	\skipday\skipday % fim de semana (sem aulas)
}


\newcommand{\semanaSegSab}{%
    \calday[Segunda-feira]{\classday}
    \skipday % Terça-feira (sem aulas)
    \skipday % Quarta-feira (sem aulas)
    \skipday % Quinta-feira (sem aulas)
    \skipday % Sexta-feira (sem aulas)
    \calday[Sábado]{\classday}
    \skipday % domingo (sem aulas)
}

\newcommand{\semanaTer}{%
 	\skipday % Segunda-feira (sem aulas)
    \calday[Terça-feira]{\classday}
    \skipday % Quarta-feira (sem aulas)
    \skipday % Quinta-feira (sem aulas)
    \skipday % Sexta-feira (sem aulas)
    \skipday\skipday % fim de semana (sem aulas)
}


\newcommand{\semanaTerQua}{%
	\skipday % Segunda-feira (sem aulas)
	\calday[Terça-feira]{\classday}
	\calday[Quarta-feira]{\classday}
	\skipday % Quinta-feira (sem aulas)
	\skipday % Sexta-feira (sem aulas)
	\skipday\skipday % fim de semana (sem aulas)
}


\newcommand{\semanaTerQui}{%
	\skipday % Segunda-feira (sem aulas)
	\calday[Terça-feira]{\classday}
	\skipday % Quarta-feira (sem aulas)
	\calday[Quinta-feira]{\classday}
	\skipday % Sexta-feira (sem aulas)
	\skipday\skipday % fim de semana (sem aulas)
}

\newcommand{\semanaTerSex}{%
	\skipday % Segunda-feira (sem aulas)
	\calday[Terça-feira]{\classday}
	\skipday % Quarta-feira (sem aulas)
	\skipday % Quinta-feira (sem aulas)
	\calday[Sexta-feira]{\classday}
	\skipday\skipday % fim de semana (sem aulas)
}

\newcommand{\semanaQua}{%
    \skipday % Segunda-feira (sem aulas)
	\skipday % Terça-feira (sem aulas)
    \calday[Quarta-feira]{\classday}
    \skipday % Quinta-feira (sem aulas)
    \skipday % Sexta-feira (sem aulas)
    \skipday\skipday % fim de semana (sem aulas)
}

\newcommand{\semanaQuaQui}{%
	\skipday % Segunda-feira (sem aulas)
	\skipday % Terça-feira (sem aulas)
	\calday[Quarta-feira]{\classday}
	\calday[Quinta-feira]{\classday}
	\skipday % Sexta-feira (sem aulas)
	\skipday\skipday % fim de semana (sem aulas)
}

\newcommand{\semanaQuaSex}{%
	\skipday % Segunda-feira (sem aulas)
	\skipday % Terça-feira (sem aulas)
	\calday[Quarta-feira]{\classday}
	\skipday % Quinta-feira (sem aulas)
	\calday[Sexta-feira]{\classday}
	\skipday\skipday % fim de semana (sem aulas)
}

\newcommand{\semanaQui}{%
    \skipday % Segunda-feira (sem aulas)
    \skipday % Terça-feira (sem aulas)
    \skipday % Quarta-feira (sem aulas)
    \calday[Quinta-feira]{\classday}
    \skipday % Sexta-feira (sem aulas)
    \skipday\skipday % fim de semana (sem aulas)
}

\newcommand{\semanaQuiSex}{%
    \skipday % Segunda-feira (sem aulas)
    \skipday % Terça-feira (sem aulas)
    \skipday % Quarta-feira (sem aulas)
    \calday[Quinta-feira]{\classday}
    \calday[Sexta-feira]{\classday}
    \skipday\skipday % fim de semana (sem aulas)
}



\newcommand{\semanaSex}{%
    \skipday % Segunda-feira (sem aulas)
    \skipday % Terça-feira (sem aulas)
    \skipday % Quarta-feira (sem aulas)
    \skipday % Quinta-feira (sem aulas)
    \calday[Sexta-feira]{\classday}
    \skipday\skipday % fim de semana (sem aulas)
}

\newcommand{\semanaSab}{%
    \skipday % Segunda-feira (sem aulas)
    \skipday % Terça-feira (sem aulas)
    \skipday % Quarta-feira (sem aulas)
    \skipday % Quinta-feira (sem aulas)
    \skipday % Sexta-feira (sem aulas)
    \calday[Sábado]{\classday}
    \skipday % domingo (sem aulas)
}

\newcommand{\semanaQuaQuiSex}{%
	\skipday % Segunda-feira (sem aulas)
	\skipday % Terça-feira (sem aulas)
	\calday[Quarta-feira]{\classday}
	\calday[Quinta-Feira]{\classday}
	\calday[Sexta-feira]{\classday}
	\skipday\skipday % fim de semana (sem aulas)
}

